\chapter{Methodology and Experimental Setup}
\section{Introduction}

%A IARA é uma plataforma robótica experimental baseada em um automóvel de passeio adaptado. Esta adaptação envolveu a instalação no automóvel de: (i) mecanismos para controlar o acelerador, freio, posição do volante, etc.; (ii) sensores; (iii) computadores para receber os dados dos sensores e controlar o automóvel; e (iv) fontes de energia para os computadores e sensores.
%Após investigar inúmeras empresas no país e no exterior, a equipe do Laboratório de Computação de Alto Desempenho (LCAD) da UFES que desenvolveu a IARA não encontrou nenhuma empresa que oferecesse tecnologia similar ou superior à oferecida pela empresa norte americana Torc Robotics (http://www.torcrobotics.com) para o acionamento dos atuadores (volante, acelerador, freio, entre outros) do automóvel e disponibilização de energia elétrica para alimentação dos computadores e sensores necessários para os estudos pretendidos com a IARA (19). À época da aquisição, a tecnologia da Torc para veículos de passeio somente operava com o automóvel Ford Escape Hybrid. Assim, o LCAD importou este automóvel (Figura 8(a)) dos USA já com as tecnologias de acionamento e disponibilização de energia instaladas pela Torc (Figura 8(b)). 
